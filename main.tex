\documentclass[sigconf,nonacm]{acmart}
\usepackage[utf8]{inputenc}
\usepackage{pgfplots}
\usepackage{subcaption}
\usepackage{adjustbox}
\usepackage{mathtools}
\usepackage{graphicx}
\usepackage{url}
%\usepackage{minted}


\DeclareUnicodeCharacter{2212}{−}
\usepgfplotslibrary{groupplots,dateplot}
\usetikzlibrary{
    patterns,
    chains,
    backgrounds,
    calc,
    shadings,
    shapes.arrows,
    arrows,
    shapes.symbols,
    shadows,
    positioning,
    decorations.markings,
    backgrounds,
    arrows.meta,
    external
}
\usepackage{array}

\pgfplotsset{compat=newest}

\newcommand{\code}[1]{\texttt{#1}}

\newif\iffinal

\iffinal
  \newcommand{\maxx}[1]{}
  \newcommand{\ryan}[1]{}
  \newcommand{\todo}[1]{}
\else
  \newcommand{\maxx}[1]{{\textcolor{red}{ Max: #1 }}}
  \newcommand{\ryan}[1]{{\textcolor{magenta}{ Ryan: #1 }}}
  \newcommand{\todo}[1]{{\textcolor{blue}{ TODO: #1 }}}
\fi


\begin{document}

\title{Ultrafast focus detection using multi-scale histologic features}


\author{Maksim Levental}
\affiliation{\institution{University of Chicago}}
\author{Ryan Chard}
\affiliation{\institution{Argonne National Laboratory}}
\author{Gregg A. Wildenberg}
\affiliation{\institution{University of Chicago}}

\begin{abstract}
    We present a fast out-of-focus detection algorithm for electron microscopy images collected serially.
    Such images are collected for the purposes of post-processing tasks such as montaging, alignment, and image segmentation.
    Such an algorithm is necessitated by recent increases in collection rates owing to advances in microscopy technology.
    Our technique adapts classical computer vision and is based on detecting various fine-grained histologic features.
    We further exploit the inherent parallelism in the technique by employing GPGPU primitives in order to accelerate characterization.
    Tests are performed that demonstrate faster than real time detection of out-of-focus conditions.
    <We also deploy to funcX something something>.
    We discuss extensions that enable scaling out to support multi-beam microscopes and integration with existing focus systems for purposes of implementing auto-focus.

\end{abstract}

\maketitle

\section{Introduction}\label{sec:intro}

Advancements in the automation of serial scanning electron microscopy (SEM)  impose a regime where thousands, if not tens of thousands, of images can now be automatically collected by researchers.
\todo{<bio use cases>}
This puts greater demand on conventional auto-focus algorithms for ensuring each image is in focus, as an alternative to the user manually evaluating each image by eye.
Without such algorithms, critical bottlenecks are created where the user is forced to reacquire individual, deficient (out-of-focus), images and manually reinsert them into the sequence of thousands of other images already acquired.
This is an onerous task which requires taking into account alignment and boundary overlap.
Furthermore, failure to quickly identify and reacquire deficient images negatively impacts the accuracy of downstream, post-processing; for example 2D montaging, 3D alignment, or automatic segmentation pipelines.
While many microscopes have builtin auto-focus algorithms, these often fail to achieve acceptable accuracy due to intrinsic mediating factors (e.g. stage drift) and extrinsic mediating factors (e.g. sample artifacts, non-uniformity in the sample).

Auto-focus technology is a critical component of many imaging systems; from consumer cameras (for purposes of convenience) to industrial inspection tools to scientific instrumentation.
Such technology is typically either active or passive; active methods exploit some auxiliary device or mechanism to measure the distance of the optics from the scene, while passive methods analyze the definition of sharpness of an image by virtue of some proxy measure.
Here we focus on passive methods, as we explicitly aim to augment existing microscopy equipment without the need for costly and complex retrofitting.

Passive proxies for the degree-of-focus (DOF) include the energy of the Laplacian, discrete cosine transform, or weighted histogram of an image; for effecting a high DOF a search can be performed.
When used as a component of an auto-focus system (as opposed to OOF detection system) all such passive methods are unsuitable for the purpose of real-time (or even near-real-time) characterization of DOF due to their long scanning times (multiple images need to be collected at potentially different depths).
As our method currently aims only to detect OOF events we do not consider or implement any focus search techniques (but do describe plans for such future work).

To overcome these challenges, thereby ensuring that images are faithfully acquired, we propose a method to evaluate image definition based multi-scale histologic feature detection (MHD).
By multi-scale histologic feature detection we mean the resolving and characterization of histological structure at multiple length scales; for our particular use-case this means structures ranging from cell walls to whole organelles.
The key insight being that the ability to resolve structure across the range of feature scales is highly correlated with a high-definition, i.e. in-focus, image.

Due to limitations of the extensibility of commercial microscopy equipment, we do not aim here to directly implement auto-focusing.
Rather than focusing the microscope, as auto-focusing algorithms would, our algorithm operates downstream of collection and reports out-of-focus (OOF) events to the user.
This enables the user to intervene and initiate reacquisition protocols (on the microscope) before unknowingly proceeding with collecting the next series of images or proceeding with downstream image processing and analysis.
This human-in-the-loop remediation protocol already saves the user much wasted collection time and tedium in triaging defective collection runs.

This rest of this article is organized as follows: section~\ref{sec:mhd} describes our focus detection method in the abstract, section~\ref{sec:implementation} discusses optimizations made in order to achieve real-time performance with our method, section~\ref{sec:evaluation} reports results of evaluating our method on sequences of images collected at varying focus depths, section~\ref{sec:related} discusses related work and how our work is distinct therefrom, and finally section~\ref{sec:conclusion} concludes with a discussion of future research.

\section{Gregg sells science!}

TYPE HERE

\section{Multi-scale Histologic Feature Detection}\label{sec:mhd}

We base our multi-scale histologic feature detection on classic on scale-space representations of signals.
We give a brief overview (a more comprehensive discussion is available~\cite{}) and describe our adaptation.

The fundamental principle of scale-space feature detection is that natural images possess structure at multiple scales and that features at a particular scale can be characterized in isolation of features at other scales.
Typically characterization is effected by convolution with a filter that satisfies the constraints of non-enhancement of local extrema, scale invariance and rotational invariance (along with some others~\cite{}).
One such filter~\cite{} is the symmetric, mean zero, 2D, Gaussian filter
$$
    G(x,y,\sigma) \coloneqq \frac{1}{2\pi \sigma^2} e^{-\frac{x^2 +y ^2}{2\sigma^2}}
$$
Thus, define the scale-space representation $L(x,y, t)$ of an image $I(x,y)$ to be the convolution of that image with a mean zero Gaussian filter:
$$
    L(x,y,t) \coloneqq G(x,y,t) * I(x,y)
$$
where $t$ is the standard deviation of the Gaussian and determines the \textit{scale} of $L(x,y,t)$.
$L(x,y,t)$ has the interpretation that image one-dimensional structures of scale smaller than $\sqrt{t^2} = t$ have been removed due to blurring.
This is due to the fact that the variance of the Gaussian filter is $t^2$ and features of this scale are therefore "beneath the noise floor" of the filter or, in effect, suppressed by filtering procedure.
A corollary is that features with length scale $t$ will have maximal response being filtered by $G(x,y,t)$; for $t' < t$ smaller length scale features will dominate the response and for $t'' > t$, as already mentioned, the response will have been suppressed.
Hence, at various scales $t$ we can use linear and non-linear combinations of space derivatives $\partial_x, \partial_y$ and derivatives in the scale $\partial_t$ to construct scale-invariant feature detectors; such feature detectors detect features such as corners, edges, and ridges.
For example, the zeros in scale of the scale normalized Laplacian
\begin{equation}
    \partial_t \nabla^2 L \coloneqq \partial_t \left( \partial_x^2 + \partial _y^2\right) L = 0
    \label{eqn:blobdetector}
\end{equation}
correspond to uniform region (otherwise known as blobs) detectors.

\section{Implementation}\label{sec:implementation}

\begin{figure}
    \centering
    \begin{subfigure}[b]{0.5\textwidth}
        \centering
        \includegraphics[width=1\linewidth]{in_focus.png}
        \caption{Histologic features of an in-focus section.}
        \label{subfig:infocus}
    \end{subfigure}
    \par\medskip
    \begin{subfigure}[b]{0.5\textwidth}
        \centering
        \includegraphics[width=1\linewidth]{out_of_focus.png}
        \caption{Histologic features of an out-of-focus section.}
        \label{subfig:outoffocus}
    \end{subfigure}
    \caption{Comparison of sections with histologic feature recognition as a function of focal depth.}
    \label{fig:histfeatsimages}
\end{figure}

We therefore propose to use a feature detector as a proxy for DOF, reasoning that quantity of features detected is positively correlated with DOF (see figure~\ref{fig:histfeatsimages}).
To this end, we develop a feature detector based on eqn.~\ref{eqn:blobdetector} but optimized for latency (rather than for accuracy).
In order to verify our hypothesis we compare the number of histologic features detected as a function of absolute deviation from in-focus ($\lvert f - f' \rvert$ where $f'$ is the correct focal depth) for a series of sections with known focal depth (see figure~\ref{subfig:degreeoofcurve}).
We observe a very strong log-linear relationship (see figure~\ref{subfig:degreeooffit}).
Fitting such a log-linear relationship produces a line with $r=-0.9754$, confirming our hypothesis that quantity of histologic features detected is a good proxy measure for DOF.

\begin{figure}
    \centering
    \begin{subfigure}[b]{0.5\textwidth}
        \centering
        % This file was created by tikzplotlib v0.9.8.
\begin{tikzpicture}[trim axis left,trim axis right]

\definecolor{color0}{rgb}{0.12156862745098,0.466666666666667,0.705882352941177}

\begin{axis}[
scaled y ticks=base 10:-3,
legend cell align={left},
legend style={fill opacity=0.8, draw opacity=1, text opacity=1, draw=white!80!black},
tick align=outside,
tick pos=left,
% title={Feature count vs. OOF},
x grid style={white!69.0196078431373!black},
xlabel={Degree of OOF},
xmin=-1.09892007559144e-06, xmax=2.31000189170211e-05,
xtick style={color=black},
xmajorgrids,
ymajorgrids,
y grid style={white!69.0196078431373!black},
ylabel={\# features},
ymin=99.9931235536055, ymax=9059.85673521721,
ytick style={color=black},
% y tick label style={/pgf/number format/sci}
]
\addplot [draw=color0, fill=color0, forget plot, mark=*, only marks]
table{%
x  y
2.89713740402389e-08 7023
1.79997717738205e-05 977
9.99997577071001e-06 1882
2.00011560321043e-06 6755
6.00014606118044e-06 4607
1.20001917481398e-05 2045
2.19998022317905e-05 743
1.99991485475958e-06 6791
3.99993008374979e-06 5606
5.99994531274e-06 3664
1.80000366866596e-05 899
2.00000519156498e-05 798
8.00016129016978e-06 3590
1.000017651916e-05 2672
1.400020697713e-05 1644
1.99997870028003e-05 874
7.99996054173021e-06 2513
1.40000062286896e-05 1233
1.60002222061202e-05 1185
1.19999909997002e-05 1419
1.60000214576702e-05 1056
4.00013083219977e-06 5610
2.20000671446296e-05 728
1.400020697713e-05 1172
3.99993008374979e-06 6308
7.99996054173021e-06 2688
9.99997577071001e-06 2037
1.03169680003984e-09 7000
1.000017651916e-05 1794
2.19998022317905e-05 741
1.99991485475958e-06 6987
5.99994531274e-06 4299
1.40000062286896e-05 1291
1.80000366866596e-05 988
2.00011560321043e-06 6389
4.00013083219977e-06 4785
6.00014606118044e-06 3080
8.00016129016978e-06 2329
1.20001917481398e-05 1472
1.60002222061202e-05 995
1.79997717738205e-05 882
1.19999909997002e-05 1595
1.60000214576702e-05 1103
2.00000519156498e-05 847
2.20000671446296e-05 781
1.99997870028003e-05 798
1.14187389609749e-07 8645
1.74120792746993e-06 8487
4.00013083219977e-06 5997
2.19998022317905e-05 831
1.99991485475958e-06 8516
3.99993008374979e-06 6280
9.99997577071001e-06 2037
1.80000366866596e-05 1004
6.00014606118044e-06 3510
8.00016129016978e-06 2658
1.20001917481398e-05 1517
1.60002222061202e-05 1095
7.99996054173021e-06 2740
2.00000519156498e-05 888
1.000017651916e-05 1958
1.79997717738205e-05 1022
1.99997870028003e-05 899
1.19999909997002e-05 1634
1.40000062286896e-05 1333
1.60000214576702e-05 1118
1.400020697713e-05 1253
5.99994531274e-06 3899
};
% \addplot [semithick, red]
% table {%
% 1.03169680003984e-09 8652.59020741432
% 2.89713740402389e-08 8621.47458132233
% 1.14187389609749e-07 8527.26131703851
% 1.74120792746993e-06 6913.51092038955
% 1.99991485475958e-06 6686.69355019204
% 1.99991485475958e-06 6686.69355019204
% 1.99991485475958e-06 6686.69355019204
% 2.00011560321043e-06 6686.52046851991
% 2.00011560321043e-06 6686.52046851991
% 3.99993008374979e-06 5166.70076818288
% 3.99993008374979e-06 5166.70076818288
% 3.99993008374979e-06 5166.70076818288
% 4.00013083219977e-06 5166.56703075379
% 4.00013083219977e-06 5166.56703075379
% 4.00013083219977e-06 5166.56703075379
% 5.99994531274e-06 3992.22674518632
% 5.99994531274e-06 3992.22674518632
% 5.99994531274e-06 3992.22674518632
% 6.00014606118044e-06 3992.1234084266
% 6.00014606118044e-06 3992.1234084266
% 6.00014606118044e-06 3992.1234084266
% 7.99996054173021e-06 3084.72952084398
% 7.99996054173021e-06 3084.72952084398
% 7.99996054173021e-06 3084.72952084398
% 8.00016129016978e-06 3084.6496741889
% 8.00016129016978e-06 3084.6496741889
% 8.00016129016978e-06 3084.6496741889
% 9.99997577071001e-06 2383.52098318378
% 9.99997577071001e-06 2383.52098318378
% 9.99997577071001e-06 2383.52098318378
% 1.000017651916e-05 2383.45928695194
% 1.000017651916e-05 2383.45928695194
% 1.000017651916e-05 2383.45928695194
% 1.19999909997002e-05 1841.70840227035
% 1.19999909997002e-05 1841.70840227035
% 1.19999909997002e-05 1841.70840227035
% 1.20001917481398e-05 1841.66073058486
% 1.20001917481398e-05 1841.66073058486
% 1.20001917481398e-05 1841.66073058486
% 1.40000062286896e-05 1423.05851843724
% 1.40000062286896e-05 1423.05851843724
% 1.40000062286896e-05 1423.05851843724
% 1.400020697713e-05 1423.02168329095
% 1.400020697713e-05 1423.02168329095
% 1.400020697713e-05 1423.02168329095
% 1.60000214576702e-05 1099.57447357174
% 1.60000214576702e-05 1099.57447357174
% 1.60000214576702e-05 1099.57447357174
% 1.60002222061202e-05 1099.54601164414
% 1.60002222061202e-05 1099.54601164414
% 1.60002222061202e-05 1099.54601164414
% 1.79997717738205e-05 849.652567006649
% 1.79997717738205e-05 849.652567006649
% 1.79997717738205e-05 849.652567006649
% 1.80000366866596e-05 849.623544825807
% 1.80000366866596e-05 849.623544825807
% 1.80000366866596e-05 849.623544825807
% 1.99997870028003e-05 656.512899491502
% 1.99997870028003e-05 656.512899491502
% 1.99997870028003e-05 656.512899491502
% 2.00000519156498e-05 656.490474517362
% 2.00000519156498e-05 656.490474517362
% 2.00000519156498e-05 656.490474517362
% 2.19998022317905e-05 507.276978773357
% 2.19998022317905e-05 507.276978773357
% 2.19998022317905e-05 507.276978773357
% 2.20000671446296e-05 507.259651356497
% 2.20000671446296e-05 507.259651356497
% };
% \addlegendentry{\#features $\approx 8653.74e^{-128941.61x}$}
\end{axis}

\end{tikzpicture}

        \caption{Number of histologic features as a function of absolute deviation from focused ($\lvert f - f' \rvert$ where $f'$ is the correct focal depth).}
        \label{subfig:degreeoofcurve}
    \end{subfigure}
    \par\medskip
    \begin{subfigure}[b]{0.5\textwidth}
        \centering
        % This file was created by tikzplotlib v0.9.8.
\begin{tikzpicture}[trim axis left,trim axis right]

\definecolor{color0}{rgb}{0.12156862745098,0.466666666666667,0.705882352941177}

\begin{axis}[
legend cell align={left},
legend style={fill opacity=0.8, draw opacity=1, text opacity=1, draw=white!80!black},
tick align=outside,
tick pos=left,
% title={Log feature count vs. OOF},
x grid style={white!69.0196078431373!black},
xlabel={Degree of OOF},
xmin=-1.09892007559144e-06, xmax=2.31000189170211e-05,
xtick style={color=black},
y grid style={white!69.0196078431373!black},
ylabel={log (\#features)},
xmajorgrids,
ymajorgrids,
ymin=6.26154869922026, ymax=9.1982215266363,
ytick style={color=black}
]
\addplot [draw=color0, fill=color0, forget plot, mark=*, mark size=1, only marks, mark options={solid,fill opacity=0}]
table{%
x  y
2.89713740402389e-08 8.85694575615902
1.79997717738205e-05 6.88448665204278
9.99997577071001e-06 7.54009032014532
2.00011560321043e-06 8.8180382503943
6.00014606118044e-06 8.43533216493592
1.20001917481398e-05 7.6231530684769
2.19998022317905e-05 6.61069604471776
1.99991485475958e-06 8.82335348511379
3.99993008374979e-06 8.63159273172473
5.99994531274e-06 8.20631072579402
1.80000366866596e-05 6.80128303447162
2.00000519156498e-05 6.68210859744981
8.00016129016978e-06 8.18590748148232
1.000017651916e-05 7.89058253465654
1.400020697713e-05 7.40488757561612
1.99997870028003e-05 6.77308037565554
7.99996054173021e-06 7.82923253754359
1.40000062286896e-05 7.11720550316434
1.60002222061202e-05 7.07749805356923
1.19999909997002e-05 7.25770767716004
1.60000214576702e-05 6.96224346426621
4.00013083219977e-06 8.63230599851674
2.20000671446296e-05 6.59030104819669
1.400020697713e-05 7.06646697013696
3.99993008374979e-06 8.74957394808293
7.99996054173021e-06 7.89655270164304
9.99997577071001e-06 7.61923341622681
1.03169680003984e-09 8.85366542803745
1.000017651916e-05 7.49220304261874
2.19998022317905e-05 6.60800062529609
1.99991485475958e-06 8.85180655855245
5.99994531274e-06 8.36613771649628
1.40000062286896e-05 7.16317239084664
1.80000366866596e-05 6.89568269774787
2.00011560321043e-06 8.76233304060234
4.00013083219977e-06 8.47324130388705
6.00014606118044e-06 8.03268487596762
8.00016129016978e-06 7.75319426988434
1.20001917481398e-05 7.29437729928882
1.60002222061202e-05 6.90274273715859
1.79997717738205e-05 6.78219205600679
1.19999909997002e-05 7.37462901521894
1.60000214576702e-05 7.0057890192535
2.00000519156498e-05 6.74170069465205
2.20000671446296e-05 6.66057514983969
1.99997870028003e-05 6.68210859744981
1.14187389609749e-07 9.06473639811739
1.74120792746993e-06 9.04629085996968
4.00013083219977e-06 8.69901462316851
2.19998022317905e-05 6.72262979485545
1.99991485475958e-06 9.04970202601337
3.99993008374979e-06 8.74512525946224
9.99997577071001e-06 7.61923341622681
1.80000366866596e-05 6.91174730025167
6.00014606118044e-06 8.16337131645991
8.00016129016978e-06 7.88532923927319
1.20001917481398e-05 7.32448997934853
1.60002222061202e-05 6.9985096422506
7.99996054173021e-06 7.91571319938212
2.00000519156498e-05 6.78897174299217
1.000017651916e-05 7.57967882309046
1.79997717738205e-05 6.92951677076365
1.99997870028003e-05 6.80128303447162
1.19999909997002e-05 7.39878627541995
1.40000062286896e-05 7.19518732017871
1.60000214576702e-05 7.01929665371504
1.400020697713e-05 7.13329595489607
5.99994531274e-06 8.2684753889826
};
\addplot [semithick, red]
table {%
1.03169680003984e-09 8.94179754399353
2.89713740402389e-08 8.93856304962636
1.14187389609749e-07 8.92869784180801
1.74120792746993e-06 8.74034245317065
1.99991485475958e-06 8.71039271209553
1.99991485475958e-06 8.71039271209553
1.99991485475958e-06 8.71039271209553
2.00011560321043e-06 8.71036947203703
2.00011560321043e-06 8.71036947203703
3.99993008374979e-06 8.47885682365952
3.99993008374979e-06 8.47885682365952
3.99993008374979e-06 8.47885682365952
4.00013083219977e-06 8.47883358360112
4.00013083219977e-06 8.47883358360112
4.00013083219977e-06 8.47883358360112
5.99994531274e-06 8.24732093522351
5.99994531274e-06 8.24732093522351
5.99994531274e-06 8.24732093522351
6.00014606118044e-06 8.24729769516622
6.00014606118044e-06 8.24729769516622
6.00014606118044e-06 8.24729769516622
7.99996054173021e-06 8.01578504678751
7.99996054173021e-06 8.01578504678751
7.99996054173021e-06 8.01578504678751
8.00016129016978e-06 8.01576180673031
8.00016129016978e-06 8.01576180673031
8.00016129016978e-06 8.01576180673031
9.99997577071001e-06 7.7842491583527
9.99997577071001e-06 7.7842491583527
9.99997577071001e-06 7.7842491583527
1.000017651916e-05 7.78422591829431
1.000017651916e-05 7.78422591829431
1.000017651916e-05 7.78422591829431
1.19999909997002e-05 7.5527132699167
1.19999909997002e-05 7.5527132699167
1.19999909997002e-05 7.5527132699167
1.20001917481398e-05 7.5526900298595
1.20001917481398e-05 7.5526900298595
1.20001917481398e-05 7.5526900298595
1.40000062286896e-05 7.32117738148079
1.40000062286896e-05 7.32117738148079
1.40000062286896e-05 7.32117738148079
1.400020697713e-05 7.3211541414235
1.400020697713e-05 7.3211541414235
1.400020697713e-05 7.3211541414235
1.60000214576702e-05 7.08964149304589
1.60000214576702e-05 7.08964149304589
1.60000214576702e-05 7.08964149304589
1.60002222061202e-05 7.08961825298749
1.60002222061202e-05 7.08961825298749
1.60002222061202e-05 7.08961825298749
1.79997717738205e-05 6.85813627279123
1.79997717738205e-05 6.85813627279123
1.79997717738205e-05 6.85813627279123
1.80000366866596e-05 6.85810560460998
1.80000366866596e-05 6.85810560460998
1.80000366866596e-05 6.85810560460998
1.99997870028003e-05 6.62660038435643
1.99997870028003e-05 6.62660038435643
1.99997870028003e-05 6.62660038435643
2.00000519156498e-05 6.62656971617397
2.00000519156498e-05 6.62656971617397
2.00000519156498e-05 6.62656971617397
2.19998022317905e-05 6.39506449592042
2.19998022317905e-05 6.39506449592042
2.19998022317905e-05 6.39506449592042
2.20000671446296e-05 6.39503382773917
2.20000671446296e-05 6.39503382773917
};
\addlegendentry{log (\#features) $\approx -115767.06x + 8.94$}
\end{axis}

\end{tikzpicture}

        \caption{Log plot and line fit with $r = -0.9754$.}
        \label{subfig:degreeooffit}
    \end{subfigure}
    \caption{Comparison of histologic feature recognition as a function of focal depth.}
    \label{fig:histfeats}
\end{figure}

We now discuss our implementation\footnote{\href{https://github.com/makslevental/cuda_blob/}{https://github.com/makslevental/cuda\_blob/}} of the feature detector, with particular attention paid to optimizations in consideration of inference latency.
Eqn.~\ref{eqn:blobdetector} permits a discretization\footnote{By virtue of $G$ being the Green's function of the heat equation $t \nabla^2 G = \partial_t G$} called \textit{Difference of Gaussians} (DoG) (see ~\cite{})
$$
    t^2 \nabla^2 L \approx  t \times \left(L(x,y, t + \delta t) - L(x,y, t)\right)
$$
Therefore, define
\begin{itemize}
    \item $\mathtt{n\_bin}$, which determines the quantity of scales determined
    \item $\mathtt{min\_t}$, the minimum scale detected
    \item $\mathtt{max\_t}$, the maximum scale detected
    \item $\delta t \coloneqq (\mathtt{max\_t} -\mathtt{min\_t})/\mathtt{n\_bin}$
    \item $t_i \coloneqq \mathtt{min\_t} + (i-1) \times \delta t$, the discrete scales detected
\end{itemize}
and finally the discretized DoG
\begin{equation}
    \operatorname{DoG}(x,y,i) \coloneqq t_i \times \left( L(x,y,t_{i+1})-L(x,y,t_i) \right)
    \label{eqn:dog}
\end{equation}
This produces a stack $\{ \operatorname{DoG}(x,y,i) \}$, in the scale dimension, of filtered and scaled images (called a Gaussian pyramid~\cite{}).

Computing the maxima of $\operatorname{DoG}(x,y,i)$ in the scale dimension (equivalently zeros of eqn.~\ref{eqn:blobdetector}) necessarily entails computing local\footnote{In a small pixel neighborhood in both space and scale dimensions.} maxima at every scale.
We make the heuristic assumption that at each pixel there is a single unique, maximal, response at some scale; this response corresponds to the scale at which the variance of the Gaussian filter $G$ most closely corresponds to the scale of the feature.
We therefore search for local maxima in $x,y$ but \textit{global} maxima in the scale dimension
\begin{equation}
    \{(\hat{x}_j, \hat{y}_j, \hat{i}_j)\} \coloneqq \operatorname*{argmaxlocal}_{x,y} \operatorname*{argmax}_{i} \operatorname{DoG}(x,y,i)
    \label{eqn:argmax}
\end{equation}
where the subscript $j$ indexes over the features detected.

It is readily apparent that our feature detector is parallelizable; for each scale $i$ we can compute $L(x,y,t_i)$ independently.
Naturally, this suggests a GPGPU implementation~\cite{}.
Therefore we develop our histologic feature detector to be maximally parallelizable in order to take advantage of the SIMT~\cite{} execution model of the conventional GPU.
A further parallelization is possible for the $\operatorname*{argmax}$ operation since the maximum is computed independently across pixels.
In order to make full use of this optimization we first perform the inner $\operatorname*{argmax}$ in eqn.~\ref{eqn:argmax} and then the outer.
The inner $\operatorname*{argmax}$ is "free", as the $\operatorname*{argmax}$ primitive is implemented in exactly this way on GPUs, and the outer $\operatorname*{argmaxlocal}$ is implemented using a $\operatorname{MaxPool2D}(n,n)$ (with $n=3$).
Employing $\operatorname{MaxPool2D}$ in this way has the added benefit of effectively performing non-maximum suppression, since it effectively rejects candidate maxima within a $3 \times 3$ neighborhood of a true maximum.

Typically one would compute  $L(x,y,t_{i})$ in the naive way (by convolving $G$ and $I$) but prior work has shown~\cite{citemerfpaper} that performing the convolution in the Fourier domain is much more efficient; namely
$$
    L(x,y,t_i) = \mathcal{F}^{-1} \big\{\mathcal{F}\{G(x,y,t_i)\} \cdot \mathcal{F}\{I(x,y)\} \big\}
$$
This approach has the added benefit that we can make use of highly optimized FFT routines made available by GPU manufacturers.
In particular we can take advantage of \textit{distributed} FFT routines; by partition the set of Gaussian filters $\{ G(x,y,t_i) \}$ across $m$ nodes we can, in principle reap, a linear increase in efficiency of the FFT.
That is to say we actually carry out
$$
    \{ L(x,y,t_i) \mid i \in I_j \} = \big\{ \mathcal{F}^{-1} \{\mathcal{F}\{G(x,y,t_i)\} \cdot \mathcal{F}\{I(x,y)\} \} \mid i \in I_j \big\}
$$
where for $j = 1, \dots, m$ the set $I_j$ indexes the scales allocated to a node $j$.
In practice FFT time (both forward and inverse) is strongly dominated by I/O but this partitioning is still crucial in instances where our images are too large to fit in the RAM available on a single GPU (see section~\ref{sec:evaluation}).

One remaining detail is histogram normalization of the images.
Due to the dynamic range (i.e. variable bit depth) of the SEM we need to normalize the histogram of pixel values; we do this by saturating $.175\%$ of the darkest pixels, saturating $.175\%$ of the lightest pixels, and mapping the entire range to $[0,1]$.
We find this gives us consistently robust results with respect to noise and anomalous features.
This histogram normalization is also parallelized using GPU primitives.

% Thus our algorithm takes the form
% \begin{minted}[escapeinside=||,mathescape=true]{python}
% def create_embedded_kernel(sigma,height,width)
%     # create (0, sigma) 2d gaussian kernel
%     # centered in array height x width

% def get_local_maxima(dogs, sigma):


% def detect_features(
%     image, n_bins, min_sigma, max_sigma
% ):
%     img_h, img_w = image.shape
%     |$\delta t$| = (max_sigma - min_sigma)/n_bins
%     sigmas = range(min_sigma, max_sigma+1, |$\delta t$|)
%     kernels = [
%         create_embedded_kernel(s, img_h, img_w)
%         for s in sigmas
%     ]
%     filtered_imgs = |$\mathcal{F}^{-1}\{ \mathcal{F}\{ \mathtt{image} \} * \mathcal{F}\{ \mathtt{kernels} \}  \}$| 
%     dog = (filtered_imgs[:-1] -    filtered_imgs[1:]) * sigmas




% \end{minted}

\section{Evaluation}\label{sec:evaluation}




\begin{figure}
    \centering
    \begin{subfigure}[b]{0.5\textwidth}
        \centering
        % This file was created by tikzplotlib v0.9.5.
\begin{tikzpicture}

\definecolor{color0}{rgb}{0.12156862745098,0.466666666666667,0.705882352941177}
\definecolor{color1}{rgb}{1,0.498039215686275,0.0549019607843137}
\definecolor{color2}{rgb}{0.172549019607843,0.627450980392157,0.172549019607843}
\definecolor{color3}{rgb}{0.83921568627451,0.152941176470588,0.156862745098039}
\definecolor{color4}{rgb}{0.580392156862745,0.403921568627451,0.741176470588235}
\definecolor{color5}{rgb}{0.549019607843137,0.337254901960784,0.294117647058824}
\definecolor{color6}{rgb}{0.890196078431372,0.466666666666667,0.76078431372549}

\begin{axis}[
legend cell align={left},
legend style={fill opacity=0.8, draw opacity=1, text opacity=1, at={(0.03,0.97)}, anchor=north west, draw=white!80!black},
tick align=outside,
tick pos=left,
x grid style={white!69.0196078431373!black},
xlabel={\(\displaystyle n\) scales},
xmajorgrids,
xmin=-0.25, xmax=49.25,
xtick style={color=black},
y grid style={white!69.0196078431373!black},
ylabel={runtime (ms)},
ymajorgrids,
ymin=19.845825, ymax=36.107675,
ytick style={color=black}
]
\addplot [semithick, color0, mark=*, mark size=1, mark options={solid,fill opacity=0}]
table {%
2 20.7355
3 20.7215
4 20.585
5 20.771
6 20.77
7 20.8295
8 20.9115
9 21.1885
10 21.1955
11 21.3655
12 21.62
13 21.5145
14 21.6065
15 21.4985
16 21.6505
17 21.588
18 21.8185
19 21.6255
20 21.9315
21 21.8115
22 21.772
23 21.9855
24 22.0175
25 22.2925
26 21.9865
27 22.379
28 22.0475
29 22.161
30 22.252
31 22.381
32 22.429
33 22.5655
34 22.634
35 22.591
36 22.466
37 22.945
38 22.7675
39 22.9475
40 22.85
41 22.6675
42 23.121
43 23.171
44 23.0435
45 23.059
46 23.3645
47 23.4815
};
\addlegendentry{1 gpu(s)}
\addplot [semithick, color1, mark=*, mark size=1, mark options={solid,fill opacity=0}]
table {%
5 23.9565
7 24.3295
9 24.701
11 25.942
13 26.0495
15 26.5605
17 27.354
19 28.2015
21 28.4405
23 28.86
25 28.729
27 30.048
29 30.642
31 31.161
33 31.5505
35 32.2785
37 32.7445
39 33.333
41 33.8215
43 34.322
45 35.3685
47 34.7745
};
\addlegendentry{2 gpu(s)}
\addplot [semithick, color2, mark=*, mark size=1, mark options={solid,fill opacity=0}]
table {%
8 22.7815
11 23.231
14 23.3275
17 24.119
20 24.936
23 25.3965
26 25.837
29 26.483
32 26.939
35 27.8365
38 28.144
41 28.6915
44 29.1745
47 29.9445
};
\addlegendentry{3 gpu(s)}
\addplot [semithick, color3, mark=*, mark size=1, mark options={solid,fill opacity=0}]
table {%
11 21.223
15 21.499
19 22.1575
23 22.62
27 23.4785
31 24.4885
35 25.425
39 26.0655
43 26.4155
47 27.315
};
\addlegendentry{4 gpu(s)}
\addplot [semithick, color4, mark=*, mark size=1, mark options={solid,fill opacity=0}]
table {%
14 23.4785
19 23.18
24 23.732
29 24.4395
34 24.8155
39 25.2365
44 26.01
};
\addlegendentry{5 gpu(s)}
\addplot [semithick, color5, mark=*, mark size=1, mark options={solid,fill opacity=0}]
table {%
17 23.415
23 23.661
29 24.188
35 24.4385
41 24.871
47 25.705
};
\addlegendentry{6 gpu(s)}
\addplot [semithick, color6, mark=*, mark size=1, mark options={solid,fill opacity=0}]
table {%
20 24.238
27 24.2995
34 25.0495
41 24.9145
};
\addlegendentry{7 gpu(s)}
\addplot [semithick, white!49.8039215686275!black, mark=*, mark size=1, mark options={solid,fill opacity=0}]
table {%
23 25.7225
31 25.5545
39 25.962
47 25.859
};
\addlegendentry{8 gpu(s)}
\end{axis}

\end{tikzpicture}

        \caption{Median runtime as a function of number of feature scales at resolution $=1024 \times 1024$.}
        \label{subfig:nbins}
    \end{subfigure}
    \par\medskip
    \begin{subfigure}[b]{0.5\textwidth}
        \centering
        % This file was created by tikzplotlib v0.9.5.
\begin{tikzpicture}

\definecolor{color0}{rgb}{0.12156862745098,0.466666666666667,0.705882352941177}
\definecolor{color1}{rgb}{1,0.498039215686275,0.0549019607843137}
\definecolor{color2}{rgb}{0.172549019607843,0.627450980392157,0.172549019607843}
\definecolor{color3}{rgb}{0.83921568627451,0.152941176470588,0.156862745098039}
\definecolor{color4}{rgb}{0.580392156862745,0.403921568627451,0.741176470588235}
\definecolor{color5}{rgb}{0.549019607843137,0.337254901960784,0.294117647058824}
\definecolor{color6}{rgb}{0.890196078431372,0.466666666666667,0.76078431372549}

\begin{axis}[
legend cell align={left},
legend style={fill opacity=0.8, draw opacity=1, text opacity=1, at={(0.03,0.97)}, anchor=north west, draw=white!80!black},
tick align=outside,
tick pos=left,
x grid style={white!69.0196078431373!black},
xlabel={resolution},
xmajorgrids,
xmin=220.8, xmax=9000.8,
xtick style={color=black},
y grid style={white!69.0196078431373!black},
ylabel={runtime (ms)},
ymajorgrids,
ymin=-7.182275, ymax=542.042775,
ytick style={color=black},
xtick={256,512,1024,2048,4096,8192},
xmode=log,
log basis x={2}
]
\addplot [semithick, color0, mark=*, mark size=1, mark options={solid,fill opacity=0}]
table {%
256 17.7825
512 18.3925
1024 21.7125
2048 36.678
4096 110.5955
8192 329.9845
};
\addlegendentry{1 gpu(s)}
\addplot [semithick, color1, mark=*, mark size=1, mark options={solid,fill opacity=0}]
table {%
256 20.0895
512 20.282
1024 26.802
2048 51.015
4096 158.818
8192 517.078
};
\addlegendentry{2 gpu(s)}
\addplot [semithick, color2, mark=*, mark size=1, mark options={solid,fill opacity=0}]
table {%
256 18.3065
512 19.7095
1024 24.269
2048 47.096
4096 131.1145
8192 470.225
};
\addlegendentry{3 gpu(s)}
\addplot [semithick, color3, mark=*, mark size=1, mark options={solid,fill opacity=0}]
table {%
256 18.196
512 19.06
1024 23.4485
2048 44.208
4096 132.185
8192 486.677
};
\addlegendentry{4 gpu(s)}
\addplot [semithick, color4, mark=*, mark size=1, mark options={solid,fill opacity=0}]
table {%
256 20.2195
512 20.115
1024 24.017
2048 45.1005
4096 118.8765
8192 499.4245
};
\addlegendentry{5 gpu(s)}
\addplot [semithick, color5, mark=*, mark size=1, mark options={solid,fill opacity=0}]
table {%
256 21.3855
512 21.679
1024 23.844
2048 43.263
4096 112.4205
8192 391.38
};
\addlegendentry{6 gpu(s)}
\addplot [semithick, color6, mark=*, mark size=1, mark options={solid,fill opacity=0}]
table {%
256 23.044
512 22.765
1024 24.8165
2048 41.827
4096 112.8525
8192 399.4225
};
\addlegendentry{7 gpu(s)}
\addplot [semithick, white!49.8039215686275!black, mark=*, mark size=1, mark options={solid,fill opacity=0}]
table {%
256 23.923
512 24.656
1024 25.793
2048 41.087
4096 120.198
8192 359.02
};
\addlegendentry{8 gpu(s)}
\end{axis}

\end{tikzpicture}

        \caption{Median runtime as a function of section resolution with 16 feature scales.}
        \label{subfig:res}
    \end{subfigure}
    \caption{Scaling experiments for runtime with respect to number of GPUs, resolution, and number of feature scales.}
    \label{fig:evalplots}
\end{figure}


Brains were prepared in the same manner and as previously described~\cite{}.
Briefly, an anesthetized animal was first transcardially perfused with 10ml 0.1 M Sodium Cacodylate (cacodylate) buffer, pH 7.4 (Electron microscopy sciences (EMS) followed by 20 ml of fixative containing 2\% paraformaldehyde (EMS), 2.5\% glutaraldehyde (EMS) in 0.1 M Sodium Cacodylate (cacodylate) buffer, pH 7.4 (EMS).
The brain was removed and placed in fixative for at least 24 hours at 4C.
A series of 300 um vibratome sections were prepared and put into fixative for 24 hours at 4C.
The primary visual cortex (V1) was identified using areal landmarks and reference atlases.
A small piece (~2 x 2 mm) containing V1 was cut out and prepared for EM by staining sequentially with 2\% osmium tetroxide (EMS) in cacodylate buffer, 2.5\% potassium ferrocyanide (Sigma-Aldrich), thiocarbohydrazide, unbuffered 2\% osmium tetroxide, 1\% uranyl acetate, and 0.66\% Aspartic acid buffered Lead (II) Nitrate with extensive rinses between each step with the exception of potassium ferrocyanide.
The tissue was then dehydrated in ethanol and propylene oxide and infiltrated with 812 Epon resin (EMS, Mixture: 49\% Embed 812, 28\% DDSA, 21\% NMA, and 2.0\% DMP 30).
The resin-infiltrated tissue was cured at 60oC for 3 days.
Using a commercial ultramicrotome (Powertome, RMC), the cured block was trimmed to a ~1.0mm x 1.5 mm rectangle and ~2,000, 40nm thick sections were collected on polyimide tape (Kapton) using an automated tape collecting device (ATUM, RMC) and assembled on silicon wafers as previously described (ref??). Images at different focal distances were acquired using backscattered electron detection with a Gemini 300 scanning electron microscope (Carl Zeiss), equipped with ATLAS software for automated imaging. Dwell times for all datasets were 1.0 microsecond.

\begin{table}
    \caption{Test platform}

    \vspace{-2ex}

    \centering
    \begin{tabular}[t]{p{0.15\linewidth}p{0.75\linewidth}}
        \hline
        CPU      & Dual AMD Rome 7742 @ 2.25GHz            \\
        GPU      & 8x NVIDIA A100-40GB                     \\
        HD       & 4x 3.84 U.2 NVMe SSD                    \\
        RAM      & 1TB                                     \\
        Software & CuPy-8.3.0, CUDA-11.0, NVIDIA-450.51.05 \\
        \hline
    \end{tabular}
    \label{tab:test}
\end{table}

We perform runtime experiments across a range of parameters of interest (section resolution, number of feature scales).
Our test platform is a NVIDIA DGX A100 (see table~\ref{tab:test}).
Experiments consist of computing the DOF of a sample section for a given configuration.
All experiments are repeated $k$ times (with $k=21$) and all metrics reported are in fact median statistics\footnote{We discard the first execution since it is an outlier due to various initializations (e.g. pinning CUDA memory).}.

For a section resolution of $1024 \times 1024$ pixels we achieve approximately a 50Hz runtime in the single GPU configuration; this is XXXX faster than real time.
We observe that, as expected, runtime grows linearly with the number of feature scales and quadratically with the resolution of the section; naturally, this is owing to the parallel architecture of the GPU.
The principle defect of our technique is that it is highly dependent on the available RAM of the GPU it is deployed to.
In practice, most GPUs available at the edge, i.e. proximal to microscopy instruments, will have insufficient ram to accommodate large section resolutions and wide feature scale ranges.
In fact, even the 40GB of the DGX's A100 is exhausted at resolutions above $4096 \times 4096$ for more than $\sim$ 20 feature scales.

\begin{figure}
    \centering
    % This file was created by tikzplotlib v0.9.5.
\begin{tikzpicture}

\definecolor{color0}{rgb}{1,1,0}

\begin{axis}[
legend cell align={left},
legend style={fill opacity=0.8, draw opacity=1, text opacity=1, at={(0.03,0.97)}, anchor=north west, draw=white!80!black},
tick align=outside,
tick pos=left,
x grid style={white!69.0196078431373!black},
xlabel={\(\displaystyle n\) bins},
xmajorgrids,
xmin=4.39, xmax=51.69,
xtick style={color=black},
y grid style={white!69.0196078431373!black},
ylabel={runtime (ms)},
ymajorgrids,
ymin=0, ymax=16.35165,
ytick style={color=black}
]
\draw[draw=black,fill=green!50.1960784313725!black] (axis cs:6.54,0) rectangle (axis cs:7.54,0.5695);
\addlegendimage{ybar,ybar legend,draw=black,fill=green!50.1960784313725!black};
\addlegendentry{Filter}

\draw[draw=black,fill=red] (axis cs:6.54,0.5695) rectangle (axis cs:7.54,1.901);
\addlegendimage{ybar,ybar legend,draw=black,fill=red};
\addlegendentry{Gather}

\draw[draw=black,fill=blue] (axis cs:6.54,1.901) rectangle (axis cs:7.54,2.053);
\addlegendimage{ybar,ybar legend,draw=black,fill=blue};
\addlegendentry{DoG}

\draw[draw=black,fill=color0] (axis cs:6.54,2.053) rectangle (axis cs:7.54,4.414);
\addlegendimage{ybar,ybar legend,draw=black,fill=color0};
\addlegendentry{Maxima}

\draw[draw=black,fill=green!50.1960784313725!black] (axis cs:8.54,0) rectangle (axis cs:9.54,0.5775);
\draw[draw=black,fill=red] (axis cs:8.54,0.5775) rectangle (axis cs:9.54,2.3175);
\draw[draw=black,fill=blue] (axis cs:8.54,2.3175) rectangle (axis cs:9.54,2.492);
\draw[draw=black,fill=color0] (axis cs:8.54,2.492) rectangle (axis cs:9.54,4.844);
\draw[draw=black,fill=green!50.1960784313725!black] (axis cs:10.54,0) rectangle (axis cs:11.54,0.599);
\draw[draw=black,fill=red] (axis cs:10.54,0.599) rectangle (axis cs:11.54,2.784);
\draw[draw=black,fill=blue] (axis cs:10.54,2.784) rectangle (axis cs:11.54,2.991);
\draw[draw=black,fill=color0] (axis cs:10.54,2.991) rectangle (axis cs:11.54,5.5385);
\draw[draw=black,fill=green!50.1960784313725!black] (axis cs:12.54,0) rectangle (axis cs:13.54,0.6445);
\draw[draw=black,fill=red] (axis cs:12.54,0.6445) rectangle (axis cs:13.54,3.279);
\draw[draw=black,fill=blue] (axis cs:12.54,3.279) rectangle (axis cs:13.54,3.519);
\draw[draw=black,fill=color0] (axis cs:12.54,3.519) rectangle (axis cs:13.54,6.091);
\draw[draw=black,fill=green!50.1960784313725!black] (axis cs:14.54,0) rectangle (axis cs:15.54,0.6885);
\draw[draw=black,fill=red] (axis cs:14.54,0.6885) rectangle (axis cs:15.54,3.739);
\draw[draw=black,fill=blue] (axis cs:14.54,3.739) rectangle (axis cs:15.54,4.0175);
\draw[draw=black,fill=color0] (axis cs:14.54,4.0175) rectangle (axis cs:15.54,6.601);
\draw[draw=black,fill=green!50.1960784313725!black] (axis cs:16.54,0) rectangle (axis cs:17.54,0.7265);
\draw[draw=black,fill=red] (axis cs:16.54,0.7265) rectangle (axis cs:17.54,4.2045);
\draw[draw=black,fill=blue] (axis cs:16.54,4.2045) rectangle (axis cs:17.54,4.517);
\draw[draw=black,fill=color0] (axis cs:16.54,4.517) rectangle (axis cs:17.54,7.0845);
\draw[draw=black,fill=green!50.1960784313725!black] (axis cs:18.54,0) rectangle (axis cs:19.54,0.762);
\draw[draw=black,fill=red] (axis cs:18.54,0.762) rectangle (axis cs:19.54,4.6885);
\draw[draw=black,fill=blue] (axis cs:18.54,4.6885) rectangle (axis cs:19.54,5.036);
\draw[draw=black,fill=color0] (axis cs:18.54,5.036) rectangle (axis cs:19.54,7.661);
\draw[draw=black,fill=green!50.1960784313725!black] (axis cs:20.54,0) rectangle (axis cs:21.54,0.801);
\draw[draw=black,fill=red] (axis cs:20.54,0.801) rectangle (axis cs:21.54,5.129);
\draw[draw=black,fill=blue] (axis cs:20.54,5.129) rectangle (axis cs:21.54,5.514);
\draw[draw=black,fill=color0] (axis cs:20.54,5.514) rectangle (axis cs:21.54,8.16);
\draw[draw=black,fill=green!50.1960784313725!black] (axis cs:22.54,0) rectangle (axis cs:23.54,0.836);
\draw[draw=black,fill=red] (axis cs:22.54,0.836) rectangle (axis cs:23.54,5.641);
\draw[draw=black,fill=blue] (axis cs:22.54,5.641) rectangle (axis cs:23.54,6.059);
\draw[draw=black,fill=color0] (axis cs:22.54,6.059) rectangle (axis cs:23.54,8.704);
\draw[draw=black,fill=green!50.1960784313725!black] (axis cs:24.54,0) rectangle (axis cs:25.54,0.8675);
\draw[draw=black,fill=red] (axis cs:24.54,0.8675) rectangle (axis cs:25.54,6.1755);
\draw[draw=black,fill=blue] (axis cs:24.54,6.1755) rectangle (axis cs:25.54,6.629);
\draw[draw=black,fill=color0] (axis cs:24.54,6.629) rectangle (axis cs:25.54,9.2745);
\draw[draw=black,fill=green!50.1960784313725!black] (axis cs:26.54,0) rectangle (axis cs:27.54,0.894);
\draw[draw=black,fill=red] (axis cs:26.54,0.894) rectangle (axis cs:27.54,6.583);
\draw[draw=black,fill=blue] (axis cs:26.54,6.583) rectangle (axis cs:27.54,7.068);
\draw[draw=black,fill=color0] (axis cs:26.54,7.068) rectangle (axis cs:27.54,9.7015);
\draw[draw=black,fill=green!50.1960784313725!black] (axis cs:28.54,0) rectangle (axis cs:29.54,0.9255);
\draw[draw=black,fill=red] (axis cs:28.54,0.9255) rectangle (axis cs:29.54,7.001);
\draw[draw=black,fill=blue] (axis cs:28.54,7.001) rectangle (axis cs:29.54,7.5195);
\draw[draw=black,fill=color0] (axis cs:28.54,7.5195) rectangle (axis cs:29.54,10.1575);
\draw[draw=black,fill=green!50.1960784313725!black] (axis cs:30.54,0) rectangle (axis cs:31.54,0.966);
\draw[draw=black,fill=red] (axis cs:30.54,0.966) rectangle (axis cs:31.54,7.6295);
\draw[draw=black,fill=blue] (axis cs:30.54,7.6295) rectangle (axis cs:31.54,8.187);
\draw[draw=black,fill=color0] (axis cs:30.54,8.187) rectangle (axis cs:31.54,10.842);
\draw[draw=black,fill=green!50.1960784313725!black] (axis cs:32.54,0) rectangle (axis cs:33.54,1.007);
\draw[draw=black,fill=red] (axis cs:32.54,1.007) rectangle (axis cs:33.54,8.0575);
\draw[draw=black,fill=blue] (axis cs:32.54,8.0575) rectangle (axis cs:33.54,8.6495);
\draw[draw=black,fill=color0] (axis cs:32.54,8.6495) rectangle (axis cs:33.54,11.308);
\draw[draw=black,fill=green!50.1960784313725!black] (axis cs:34.54,0) rectangle (axis cs:35.54,1.0335);
\draw[draw=black,fill=red] (axis cs:34.54,1.0335) rectangle (axis cs:35.54,8.4875);
\draw[draw=black,fill=blue] (axis cs:34.54,8.4875) rectangle (axis cs:35.54,9.1125);
\draw[draw=black,fill=color0] (axis cs:34.54,9.1125) rectangle (axis cs:35.54,11.8175);
\draw[draw=black,fill=green!50.1960784313725!black] (axis cs:36.54,0) rectangle (axis cs:37.54,1.08);
\draw[draw=black,fill=red] (axis cs:36.54,1.08) rectangle (axis cs:37.54,9.0505);
\draw[draw=black,fill=blue] (axis cs:36.54,9.0505) rectangle (axis cs:37.54,9.7105);
\draw[draw=black,fill=color0] (axis cs:36.54,9.7105) rectangle (axis cs:37.54,12.442);
\draw[draw=black,fill=green!50.1960784313725!black] (axis cs:38.54,0) rectangle (axis cs:39.54,1.1055);
\draw[draw=black,fill=red] (axis cs:38.54,1.1055) rectangle (axis cs:39.54,9.563);
\draw[draw=black,fill=blue] (axis cs:38.54,9.563) rectangle (axis cs:39.54,10.26);
\draw[draw=black,fill=color0] (axis cs:38.54,10.26) rectangle (axis cs:39.54,12.9845);
\draw[draw=black,fill=green!50.1960784313725!black] (axis cs:40.54,0) rectangle (axis cs:41.54,1.134);
\draw[draw=black,fill=red] (axis cs:40.54,1.134) rectangle (axis cs:41.54,10.048);
\draw[draw=black,fill=blue] (axis cs:40.54,10.048) rectangle (axis cs:41.54,10.775);
\draw[draw=black,fill=color0] (axis cs:40.54,10.775) rectangle (axis cs:41.54,13.509);
\draw[draw=black,fill=green!50.1960784313725!black] (axis cs:42.54,0) rectangle (axis cs:43.54,1.1985);
\draw[draw=black,fill=red] (axis cs:42.54,1.1985) rectangle (axis cs:43.54,10.5625);
\draw[draw=black,fill=blue] (axis cs:42.54,10.5625) rectangle (axis cs:43.54,11.331);
\draw[draw=black,fill=color0] (axis cs:42.54,11.331) rectangle (axis cs:43.54,14.085);
\draw[draw=black,fill=green!50.1960784313725!black] (axis cs:44.54,0) rectangle (axis cs:45.54,1.199);
\draw[draw=black,fill=red] (axis cs:44.54,1.199) rectangle (axis cs:45.54,10.988);
\draw[draw=black,fill=blue] (axis cs:44.54,10.988) rectangle (axis cs:45.54,11.7865);
\draw[draw=black,fill=color0] (axis cs:44.54,11.7865) rectangle (axis cs:45.54,14.5375);
\draw[draw=black,fill=green!50.1960784313725!black] (axis cs:46.54,0) rectangle (axis cs:47.54,1.2395);
\draw[draw=black,fill=red] (axis cs:46.54,1.2395) rectangle (axis cs:47.54,11.396);
\draw[draw=black,fill=blue] (axis cs:46.54,11.396) rectangle (axis cs:47.54,12.227);
\draw[draw=black,fill=color0] (axis cs:46.54,12.227) rectangle (axis cs:47.54,14.9815);
\draw[draw=black,fill=green!50.1960784313725!black] (axis cs:48.54,0) rectangle (axis cs:49.54,1.27);
\draw[draw=black,fill=red] (axis cs:48.54,1.27) rectangle (axis cs:49.54,11.96);
\draw[draw=black,fill=blue] (axis cs:48.54,11.96) rectangle (axis cs:49.54,12.8295);
\draw[draw=black,fill=color0] (axis cs:48.54,12.8295) rectangle (axis cs:49.54,15.573);
\end{axis}

\end{tikzpicture}

    \caption{Breakdown of runtime into the four major phases for two GPUs across feature scales at resolution $=1024 \times 1024$.}
    \label{fig:stacked}
\end{figure}

Therefore, we further investigate parallelizing MHD across multiple GPUs.
Our implementation parallelizes MHD in a straightforward fashion: we partition the set of filters across the GPUs, perform the ``lighter'' FFT-IFFT pair on each constituent GPU, and then gather the results to the root GPU (arbitrarily chosen).
Note that for such multi-GPU configurations the range of feature scales was chosen to be a multiple of the number of GPUs (hence the proportionally increasing sparsity of data in figure~\ref{subfig:nbins}).
We observe that, as one would expect, runtime is inversely proportional to number of GPUs (see figure~\ref{subfig:res}) but that for instances where a single GPU configuration is sufficient it is also optimal.
More precise timing reveals that parallelization across multiple GPUs incurs high copy costs during the gather phase of parallel MHD (see figure~\ref{fig:stacked}).
Note that this latency persists even after taking advantage of CUDA IPC~\cite{6270863}.
In effect, this is a fairly obvious demonstration of Amdahl's law.
Therefore, we emphasize that parallelization across multiple GPUs should only be considered in instances where full resolution section images are of the utmost necessity\footnote{For example, when feature scale range are very wide, with detection at the lower end of the scale being critical. In all other cases downsampling by bilinear interpolation in order to satisfy GPU RAM constraints yields a more than reasonable tradeoff between accuracy and latency.}.

\section{Related work}\label{sec:related}
\section{Conclusion}\label{sec:conclusion}

\begin{acks}
    This work was supported by the U.S. Department of Energy, Office of Science, under contract DE-AC02-06CH11357.
    % \ryan{Marius LDRD}
\end{acks}

\bibliographystyle{ACM-Reference-Format}
\bibliography{biblio}

\end{document}


